%% General syllabus https://docs.google.com/document/d/1E2sHqv5dZ0lZsQYAIUxpZgtufwUmfEhPNOVYi0tOl7M/edit#

% https://www.jstor.org/stable/2709374?seq=1
% https://onlinelibrary.wiley.com/doi/full/10.1111/j.1469-1809.2011.00649.x
% https://twitter.com/Graham_Coop/status/941164263803985921
%https://link.springer.com/article/10.1023/A:1018396529332

\section*{Genetics, Eugenics, and Scientific Racism.}
\marginnote{This is a complex historical topic, with many geneticists adopting different, sometimes conflicting positions over their lifetimes as their views and those of society changed. I cite both the primary genetics literature and historical analysis (where available).}
The history of genetics and evolutionary biology is intertwined with the history of eugenics and scientific racism. Francis Galton, one of the first people to systematically study human inheritance, coined the term “eugenics” in 1883 to describe the idea of `human improvement' through controlled breeding of humans \citep{galton1883inquiries}. Historically, eugenics is much more than just the idea that selection through breeding would work in humans; it is the idea that particular people are “genetically inferior” and therefore “unfit” to reproduce \citep{paul2014wrong}.  Eugenicists’ obsession with human worth and genetic inferiority also meant that eugenicists also often held that people from some races and ethnicities are genetically superior to others. Thus, ideas about eugenics also built on older racist fields of science that sought to classify humans into a discrete racial hierarchy, while in parallel scientists in these fields were forcing ideas from genetics and evolution into an essentialist view of race. These deeply flawed hierarchies have frequently been used by the powerful to justify subjugating and disenfranchising minorities and Indigenous people. 

 Although eugenics is often correctly associated with the Nazi party and the Holocaust, eugenic ideas and eugenic policies were also widespread in the US and UK during the 1920s and 1930s and sometimes aligned with progressive causes of that era \citep{paul1984eugenics,kevles1995name}. Eugenic ideas were also implemented as policy ---with horrific consequences---in a number of countries. Immigration policies based explicitly on eugenic arguments were put in place in the US from the 1920s until their repeal in the 1950s and 60s. These policies strongly favoured immigration from Northern Europe and were a deliberate action to restrict or bar immigration from Asia and eastern and southern Europe based on xenophobic, racist, and anti-Semitic views \citep{okrent2020guarded}. During the 20th Century, many US states passed eugenics sterilization laws  \citep{reilly2015eugenics}, that in practice were often targeted against Black, Latino, and Indigenous people \citep{hansen2013sterilized}. For example, the state of California from 1919 to 1972 used eugenics ideas to justify the sterilization of 20,000 people who had been labelled unfit and mentally defective, a disproportionate number of whom were Latino \citep{stern2017california,novak2018disproportionate} 
 
%https://www.theatlantic.com/health/archive/2017/01/california-sterilization-records/511718/
%https://www.ncbi.nlm.nih.gov/pmc/articles/PMC5888070/
% https://www.ncbi.nlm.nih.gov/pmc/articles/PMC5308144/
%state map  https://www.pbs.org/independentlens/blog/unwanted-sterilization-and-eugenics-programs-in-the-united-states/
%https://www.zocalopublicsquare.org/2016/01/06/when-california-sterilized-20000-of-its-citizens/chronicles/who-we-were/

Many early geneticists during this time were proponents of eugenics and many supported racist views in their genetics research. One notable example is R.A. Fisher, who we'll encounter throughout this book. Fisher is arguably the father of much of evolutionary genetics and modern statistics, having made huge contributions to the foundations of both fields. He pursued these fields in part because of his eugenic interests and concerns about the ``genetically inferiority” of the lower classes \citep{norton1983fisher,mazumdar2005eugenics_fisher}. For example, he devoted a number of the later chapters in his classic evolutionary genetics book to eugenics \citep{fisher1930}. He was hardly alone in his views, with many prominent geneticists lending their voices to eugenic and racist arguments. Indeed, many famous genetics institutions grew from roots in eugenics. For example, the Cold Spring Harbor Laboratory hosted a large Eugenics Record Office, and prior to 1954, the journal Annals of Human Genetics was called Annals of Eugenics. Scientists and their institutions strongly shaped the eugenic views and policies of their time and at times bent science to lend support to their racist views. Given their lasting contributions to our field, we should not shy away from reading and discussing their work. But despite their scientific accomplishments, we should resist the urge to celebrate or idolize them. We should also guard against inheriting their thinking by continually questioning the frameworks and language they put in place.
%https://sci-hub.tw/https://www.annualreviews.org/doi/10.1146/annurev-genom-090314-024930?url_ver=Z39.88-2003&rfr_id=ori%3Arid%3Acrossref.org&rfr_dat=cr_pub++0pubmed
% https://www.gwern.net/docs/statistics/1981-mackenzie-statisticsinbritain18651930.pdf

From its inception, geneticists have also been central to movements against eugenics and scientific racism on scientific as well as moral grounds. For instance, Thomas Hunt Morgan and Lancelot Hogben were both prominent geneticists who argued that eugenicists failed to recognize the environmental and social causes of inequality \citep{hogben1933nature,tabery2008ra,allen2011eugenics}. These arguments thread into later debates, where geneticists pushed back on simplistic and erroneous claims about genetics, IQ and behavioural differences among human populations \citep{dobzhansky1961bogus,lewontin1970race,paul1994dobzhansky}. Population geneticists have also been central to the pushback against scientific racism, highlighting the close genetic relationships among all humans due to their recent common ancestry and the ephemeral nature of populations \citep{united1952race,lewontin1972apportionment,provine1986geneticists,gannett2013theodosius}. Racists continue to advance a selective view of population-genetic results to further their ends. As scientists, it is too easy to claim that we are just interested in the facts and ignore others who seek to present a distorted view of the science to advance their own political and social agendas. It is our job as population geneticists to argue against misuse of our field. As human genomics and personal genomics rise in prominence, we also need to resist public adoption of genetic determinism and essentialist, racialized thinking. We must question the topics we choose to investigate, the assumptions we make, and the conversations we prioritize as a field. Through exploring our own biases and those embedded in the presentation and use of our field, we can help to combat the misrepresentations of genetics and evolution that continue to cause harm in our society.
% Dobzhansky evolving  https://sci-hub.tw/https://princetonup.degruyter.com/view/book/9781400863808/10.1515/9781400863808.219.xml
%https://sci-hub.tw/https://academic.oup.com/jhered/article-abstract/52/4/189/812767
