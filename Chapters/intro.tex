\section*{Introduction}

Evolution is change over time. Biological evolution is change over time in the
genetic composition of populations. The genetic composition of a
population is the set of genomes that the individuals in our population
carry. While at first this definition of evolution seems at odds with the
common textbook view of the evolution of phenotypes (such as the changing shape
of the finch beaks over generations) it is genetic changes that underpin these
phenotypic changes.  \\

The genetic composition of the population can
alter due to the death of individuals or the migration of individuals in or out
of the population. If our individuals have different numbers of children, this
also alters the genetic composition of the population in the next generation.
Every new individual born into the population subtly changes the genetic
composition of the population. Their genome is a unique combination of their
parents' genomes, having been shuffled by segregation and recombination during
meioses, and possibly changed by mutation. \\

Population genetics is the study of the genetic composition of natural
populations. It seeks to understand how this composition has been changed over
time by the forces of mutation, recombination, selection, migration, and
genetic drift.  To understand how these forces interact, it is helpful to
develop simple theoretical models to help our intuition. In these notes we will
work through these models and summarize the major areas of population genetic
theory. While these models will seem na\"{\i}ve (and indeed they are) they are
nonetheless incredibly useful and powerful. Throughout the course we will see
that these simple models yield accurate predictions, such that much of our
understanding of the process of evolution is built on these models. We will
also see how these models are incredibly useful for understanding real patterns
we see in the evolution of phenotypes and genomes, such that much of our
analysis of evolution, in a range of areas from human genetics to conservation,
is based on these models. Therefore, population genetics is key to
understanding various applied questions from how medical genetics identifies
the genes involved in disease to how we preserve small populations (such as a
Florida panther) from extinction. 



``Dobzhansky (1951) \cite{DobzhanskyBook} once defined evolution as 'a change in the genetic
composition of the populations' (p. 16) an epigram that should not be
mistaken for the claim that everything worth saying about evolution is
contained in statements about genes'' \citet{lewontin01} 


\newpage