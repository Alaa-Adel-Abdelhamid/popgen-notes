\chapter{Introduction}
\newthought{Evolution is change over time.} Biological evolution is the change over time in the genetic composition of a population.\cite{DobzhanskyBook} Our population is made up of a set of interbreeding individuals, the genetic composition which is made up of the  genomes that each individual carries.  While at first this definition of evolution seems at odds with the
common textbook view of the evolution of phenotypes (such as the changing shape
of the finch beaks over generations) it is genetic changes that underpin these
phenotypic changes.  \\

The genetic composition of the population can alter due to the death of individuals or the migration of individuals in or out
of the population. If our individuals have different numbers of children, this
also alters the genetic composition of the population in the next generation.
Every new individual born into the population subtly changes the genetic
composition of the population. Their genome is a unique combination of their
parents' genomes, having been shuffled by segregation and recombination during
meioses, and possibly changed by mutation. \\

Population genetics is the study of the genetic composition of natural
populations. It seeks to understand how this composition has been changed over
time by the forces of mutation, recombination, selection, migration, and
genetic drift.  To understand how these forces interact, it is helpful to
develop simple theoretical models to help our intuition. In these notes we will
work through these models and summarize the major areas of population genetic
theory.


\marginnote{``All models are wrong but some are useful'' -
  \citeauthor{box:79} (1979).}  % https://en.wikipedia.org/wiki/All_models_are_wrong#Quotations_of_George_Box
% http://www.dtic.mil/dtic/tr/fulltext/u2/a070213.pdf

While the population genetic models we will develop will seem na\"{\i}ve (and indeed they are) they are
nonetheless incredibly useful and powerful. Throughout the course we will see
that these simple models yield accurate predictions, such that much of our
understanding of the process of evolution is built on these models. We will
also see how these models are incredibly useful for understanding real patterns
we see in the evolution of phenotypes and genomes, such that much of our
analysis of evolution, in a range of areas from human genetics to conservation,
is based on these models. Therefore, population genetics is key to
understanding various applied questions from how medical genetics identifies
the genes involved in disease to how we preserve species from extinction. 

\newthought{Population genetics emerged} from early efforts to
reconcile Mendelian genetics with Darwinian thought.\marginnote{See
\citet{provine:01} \cite{provine:01} for a history of early population
genetics.}
Part of the power of
population genetics comes from the fact that the basic rules of
transmission genetics are simple and nearly universal.  One of the truely remarkable things about population genetics is that
many of the important ideas and mathematical models emerged before the
1940s, long before the
mechanistic-basis of inheritance (DNA) was discovered, and yet the
usefulness of these models has not diminished. This is a testiment to
the fact that the models are established on a very solid foundation,
building from the basic rules of genetic transmission combined with
simple mathematical and statistical models.   

\paragraph{Population genetics is a necessary but not sufficient description of evolution.}
\begin{quote}
``Dobzhansky (1951) once defined evolution as 'a change in the genetic
composition of the populations' (p. 16) an epigram that should not be
mistaken for the claim that everything worth saying about evolution is
contained in statements about genes'' -Lewontin \cite{lewontin01} 
\end{quote}


%Population geneticists can sometimes seem
%myopic in their focus on the nuts and bolts of evolution. The gradeur
%of evolution can sometimes seem a little lost in the simple models of
%population geneticists and in the field's obsession with genomic
%sequencing. 

%The fact that many evolutionary principals are underpinned by
%population genetic models does diminish other
%areas of evolutionary study. 

%We certainly do not need to know all of the genes underlying the 
%displays of the bird of paradise's to study how the divergence of
%these displays due to sexual selection may drive speciation. 
%Nor do we need know the precise selection pressures that drove
%to study the developmental mechanisms XXXX. 
%Many of these aspects of any given things may be fundementally
%unknowable, and may often not be of primary interest to a given field
%of study. 

%It is also not true that 
\newpage