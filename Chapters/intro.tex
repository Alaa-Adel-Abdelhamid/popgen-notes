\chapter{Introduction}
\newthought{Evolution is change over time.} Biological evolution is the change over time in the genetic composition of a population.\cite{DobzhanskyBook} Our population is made up of a set of interbreeding individuals, the genetic composition of which is made up of the  genomes that each individual carries.  While at first this definition of evolution seems at odds with the
common textbook view of the evolution of phenotypes, such as the changing shape
of finch beaks over generations, it is genetic changes that underpin these
phenotypic changes.   \\

The genetic composition of the population can alter due to the death of individuals or the migration of individuals in or out
of the population. If our individuals vary in the number of children they have, this
also alters the genetic composition of the population in the next generation.
Every new individual born into the population subtly changes the genetic
composition of the population. Their genome is a unique combination of their
parents' genomes, having been shuffled by segregation and recombination during
meioses, and possibly changed by mutation. These individual events seem minor at the level of the population, but it is the accumulation of small changes in aggregate across individuals and generations that is the stuff of evolution. It is the compounding of these small changes over tens, hundreds, and millions of generations that drives the amazing diversity of life that has emerged on this earth.\\

Population genetics is the study of the genetic composition of natural
populations and its evolutionary causes and consequences. Quantitative genetics is the study of the genetic basis of phenotypic variation and how phenotypic changes can evolve. Both fields are closely conceptually aligned as we'll see throughout these notes. They seek to describe how the genetic and phenotypic composition of populations can be changed over
time by the forces of mutation, recombination, selection, migration, and
genetic drift.  To understand how these forces interact, it is helpful to
develop simple theoretical models to help our intuition. In these notes we will
work through these models and summarize the major areas of population- and quantitative-genetic
theory.\\


\marginnote{``All models are wrong but some are useful'' -
  \citeauthor{box:79} (1979).}  % https://en.wikipedia.org/wiki/All_models_are_wrong#Quotations_of_George_Box
% http://www.dtic.mil/dtic/tr/fulltext/u2/a070213.pdf

While the models we will develop will seem na\"{\i}ve, and indeed they are, they are
nonetheless incredibly useful and powerful. Throughout the course we will see
that these simple models often yield accurate predictions, such that much of our
understanding of the process of evolution is built on these models. We will
also see how these models are incredibly useful for understanding real patterns
we see in the evolution of phenotypes and genomes, such that much of our
analysis of evolution, in a range of areas from human medical genetics to conservation,
is based on these models. Therefore, population and quantitative genetics are key to
understanding various applied questions, from how medical genetics identifies
the genes involved in disease to how we preserve species from extinction. \\

Population genetics emerged from early efforts to
reconcile Mendelian genetics with Darwinian thought.\marginnote{See
\citet{provine:01} \cite{provine:01} for a history of early population
genetics.}
Part of the power of
population genetics comes from the fact that the basic rules of
transmission genetics are simple and nearly universal.  One of the truly remarkable things about population genetics is that
many of the important ideas and mathematical models emerged before the
1940s, long before the
mechanistic-basis of inheritance (DNA) was discovered, and yet the
usefulness of these models has not diminished. This is a testament to
the fact that the models are established on a very solid foundation,
building from the basic rules of genetic transmission combined with
simple mathematical and statistical models.   

Much of this early work traces to the ideas of R.A. Fisher, Sewall Wright, and J.B.S. Haldane, who, along with many others, described the early principals and mathematical models underlying our understanding of the evolution of populations. Building on this conceptual fusion of genetics and evolution, there followed a flourishing of evolutionary thought, the modern evolutionary synthesis, combining these ideas with those from the study of speciation, biodiversity, and paleontology. In total this work showed that both short-term evolutionary change and the long-term evolution of biodiversity could be well understood through the gradual accumulation of evolutionary change within and among populations. This evolutionary synthesis continues to this day, combining new insights from genomics, phylogenetics, ecology, and developmental biology.

Population and quantitative genetics are a necessary but not a sufficient description of evolution; it is only by combining the insights of many fields that a rich and comprehensive picture of evolution emerges.
\marginnote{
\begin{quote}
``\citet{DobzhanskyBook} once defined evolution as 'a change in the genetic
composition of the populations' an epigram that should not be
mistaken for the claim that everything worth saying about evolution is
contained in statements about genes'' 
\end{quote} -- \citeauthor{lewontin01} }
We certainly do not need to know the genes underlying the displays of the birds of paradise to study how the divergence of these displays, due to sexual selection, may drive speciation. Indeed, as we'll see in our discussion of quantitative genetics, we can predict how populations respond to selection, including sexual selection and assortative mating, without any knowledge of the loci involved. Nor do we need to know the precise selection pressures and the ordering of genetic changes to study the emergence of the tetrapod body plan. We do not necessarily need to know all the genetic details to appreciate the beauty of these, and many other, evolutionary case-studies. \ec{However, every student of biology gains from understanding the basics of population and quantitative genetics that allow us to base our evolutionary studies and speculations on a solid bedrock of understanding of the processes that underpin all evolutionary change.}

%A full description of evolution only emerges from an understanding of the 


%\ec{this is an awkward segway}
%Population geneticists can sometimes seem
%myopic in their focus on the nuts and bolts of evolution. The grandeur
%of evolution can sometimes seem a little lost in the simple models of
%population geneticists and in the field's obsession with genomic
%sequencing. 

%The fact that many evolutionary principals are underpinned by
%population genetic models does diminish other
%areas of evolutionary study. 

%We certainly do not need to know all of the genes underlying the 
%displays of the bird of paradise's to study how the divergence of
%these displays due to sexual selection may drive speciation. 
%Nor do we need know the precise selection pressures that drove
%to study the developmental mechanisms XXXX. 
%Many of these aspects of any given things may be fundementally
%unknowable, and may often not be of primary interest to a given field
%of study. 

%It is also not true that 