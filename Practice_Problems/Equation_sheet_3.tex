\documentclass[12pt,twocolumn]{article}
\usepackage{nicefrac}
 \usepackage{amsmath}
  \usepackage{amsfonts}
  \usepackage{amssymb}
\newcounter{question}[section]   %%modified from https://www.sharelatex.com/learn/Counters
%\newenvironment{question}[1][]{\refstepcounter{question}\par   \begin{tcolorbox}
 %   \medskip \textbf{Question~\thequestion. #1}\rmfamily}{\medskip} \end{tcolorbox}
\newcommand{\E}{\mathbb{E}}
\renewcommand{\P}{\mathbb{P}}
\newcommand{\half}{\tfrac{1}{2}}

\newcommand{\wbar}{\overline{w}}
% New commands added by Simon:
\newcommand{\fis}{F_{\mathrm{IS}}}
\newcommand{\fit}{F_{\mathrm{IT}}}
\newcommand{\fst}{F_{\mathrm{ST}}}
\newcommand{\Wbar}{\overline{W}}

\newenvironment{question}[1][]{\refstepcounter{question}\par\medskip
   \textbf{Question~\thequestion. #1} \rmfamily}{\medskip} 


\begin{document}
%\section*{EVE102 Equation Sheet for Midterm 3}

\begin{center}
\begin{tabular}{ccc}
$(1-F) p^2 + F p$, & $(1-F) 2pq$, & $(1-F) q^2 + F q$ \nonumber
\end{tabular}
\end{center}

\begin{equation}
q_{eq} = \frac{s_1}{s_1+s_2} \nonumber
\end{equation}

\begin{equation}
D_{AB} = p_{AB} - p_Ap_B \nonumber
\end{equation}

\begin{equation}
	\frac{\sqrt[t]{\prod_{i=0}^{t-1}w_{12,i}}}{\sqrt[t]{\prod_{i=0}^{t-1}w_{22,i}}} \nonumber
\end{equation}

\begin{equation}
  H_t = \left(1-\frac{1}{2N_e} \right)^tH_0 \approx H_0 e^{-\nicefrac{t}{2N_e}} \nonumber
\end{equation}

\begin{equation}
  D_t=  (1-r)^t D_0 \approx D_0 e^{-rt} \nonumber
\end{equation}

\begin{equation}
\E[T_k] = \frac{2 N_e}{ {k \choose 2} },~~~~\E[T_{MRCA}] =
4N_e(1-1/n) \nonumber
\end{equation}

\begin{equation}
\E[T_{tot}] = \sum_{k=n}^2 \frac{4N_e}{k-1}  \nonumber
\end{equation}

\begin{equation}
\ell^* = \frac{-\log(0.5)}{r_{BP} \tau }  \nonumber
\end{equation}

\begin{equation}
  H = \frac{4N_e\mu}{1+4N_e\mu} \approx 4N_e\mu  \nonumber
\end{equation}

\begin{equation}
  F_{ij}= 0 \times r_0 + (\nicefrac{1}{4}) r_1  + (\nicefrac{1}{2}) r_2.  \nonumber
\label{eqn:coeffkinship}
\end{equation}

\begin{equation}
R = h^2 S = V_A \beta \nonumber
\end{equation}


\begin{equation}
(1-\fit) =\frac{H_I}{H_S} \frac{H_S}{H_T}=(1-\fis)(1-\fst).\nonumber
\label{eqn:F_relationships}
\end{equation}

\begin{equation}
\fit =1-\frac{H_I}{H_T},~~\fis =1-\frac{H_I}{H_S}.\nonumber
%\label{eqn:F_relationships}
\end{equation}

\begin{equation}
\fst =1-\frac{H_S}{H_T}.\nonumber
%\label{eqn:F_relationships}
\end{equation}

\begin{equation}
  \fst = \frac{ T}{ T + 4N_e } \nonumber
%\label{eqn:F_relationships}
\end{equation}

\begin{equation}
  F_{IM} = \frac{1}{1 + 4N_I m} \nonumber
%\label{eqn:F_relationships}
\end{equation}


\begin{equation}
	\tau = \frac{2}{s} \log \left(\frac{p_{\tau} q_0}{q_{\tau}
            p_0}\right) \nonumber
\end{equation}

\begin{equation}
\overline{w} = w_{11}p^2+w_{12}2pq+w_{22}q^2  \nonumber
\end{equation}

\begin{equation}
q_{eq} = \frac{\mu}{hs}  \nonumber
\end{equation}
\begin{equation}
q_{eq} =\sqrt{\frac{\mu}{s}}~~~\textrm{if } h=0 \nonumber
\end{equation}

\begin{equation}
Cov(X_1,X_2)  = 2 F_{1,2} V_A  \nonumber
\end{equation}

\begin{equation}
Cov(X_1,X_2)  = 2F_{1,2} V_A + r_2 V_D~~~\textrm{if } V_D>0  \nonumber
\end{equation}

\begin{equation}
\pi \left(\frac{1}{2N} \right) = \frac{1-e^{-s }}{1-e^{-2Ns}}  \nonumber
\end{equation}

\end{document}
