\documentclass[12pt]{article}
\usepackage{nicefrac}
\newcounter{question}[section]   %%modified from https://www.sharelatex.com/learn/Counters
%\newenvironment{question}[1][]{\refstepcounter{question}\par   \begin{tcolorbox}
 %   \medskip \textbf{Question~\thequestion. #1}\rmfamily}{\medskip} \end{tcolorbox}


\newenvironment{question}[1][]{\refstepcounter{question}\par\medskip
   \textbf{Question~\thequestion. #1} \rmfamily}{\medskip} 


\begin{document}
\title{EVE102 Practice Problems 1}
\author{Graham Coop}
\date{}
\maketitle

{\bf Unless otherwise stated assume that loci are biallelic and
autosomal. Show your work, and state the asumptions you have to make
(if any). }

\begin{question}
You are studying a codominant flower colour polymorphism. Skipping
through a meadow of flowers you compile the following data:
\begin{center}
\begin{tabular}{ccc}
red & pink  & white\\
200 & 100 & 200\\
\end{tabular}
\end{center}
{\bf A)} What frequencies would you expect at this locu under Hardy Weinberg equilibrium? \\
{\bf B)} Calculate the inbreeding coefficient at this locus.\\ 
{\bf C)} Name two distinct processes that could lead to the deviation
you see, and describe how they would result in a deficit of heterozygotes.
\end{question}

\begin{question}
The colour and shape of a a species of beetles wings are controled by two distinct
polymorphisms (with alleles big/small and red/yellow respectively). 
In a museum collection you estimate the frequency of the four
haplotypes to be:\\
\begin{center}
\begin{tabular}{cccc}
big/red & big/yellow & small/red & small/yellow\\
0.69 & 0.00 & 0.09 & 0.22\\
\end{tabular}
\end{center}
This collection is from 60 years ago. In present day populations you
estimate the frequencies of the haplotypes to be:\\
\begin{center}
\begin{tabular}{cccc}
0.5452 & 0.1448 & 0.2348 & 0.0752\\
\end{tabular}
\end{center}
{\bf A)} Assuming one generation per year, what is the recombination fraction
between these loci?
{\bf B} Qualitatively how would your answer change if you determined
that crossing over only occurred in females and not in males?
\end{question}

\begin{question}
Charles the 2$^{nd}$ of Spain was the offspring of a uncle-niece marriage. 
{\bf A)} What were the relatedness coefficients ($r_0$, $r_1$, $r_2$) of his parents? 
{\bf B)} What was Charles's inbreeding coefficient from this inbreeding loop?
In answering this question ignore the other deeper inbreeding in his pedigree.
\end{question}

\begin{question}
What are the relatedness coefficients of the X chromosome between:\\
1) Two male full siblings?\\
2) Two female full siblings?\\
\end{question}

\begin{question}
You are studying the wing spot polymorphism in a butterfly
species. From crosses in the lab you find that the presence of
wing spots is determined by a dominant allele.\\
You collect 100 butterflies, 84 of them have the wing spots. What is
the frequency of the wing-spot allele?  
\end{question}

\begin{question}
An allele has frequency of $0.01$ in the population. What the
probability that both you and your first (full) cousin are heterozygote for
the alele?
\end{question}

\begin{question}
The kinship coefficient of the parents is the inbreeding coefficient
of the offspring. Explain, with reference to the weighting of
relatedness coefficients in the inbreeding coefficient, why the
inbreeding coefficient is the probability that a locus is homozygous
by descent.
\end{question}

\begin{question}
Explain why multiple inbreeding loops in an individual's pedigree
lead to higher levels of inbreeding. 
\end{question}

\begin{question}
Based on museum samples, from $\sim 1800$, you estimate that the
average heterozygosity in Northern Elephant Seals
was $0.0304$ across many loci. Based on futher samples you estimate
that in $1960$ this had dropped to $0.011$. Elephant Seals have a
generation time of $8$ years. \\
What effective population size do you estimate is consistent with this
drop? 
\end{question}


\begin{question}
Assume that at a locus, where allele 1 has frequency $p$. What is
the probability that two 1/2 sibs are both heterozygotes? 
\end{question}


\begin{question}
{\bf A)} Why are large populations expected to harbor more neutral variation?\\
{\bf B)} What is the effective population size? Is it usually higher or lower than the census population size?
\end{question}

\begin{question}
You are providing genetic counsoling for a couple. They are first
cousins. The man is a heterozgote carrier for a rare recessive disease
allele, the carrier status for the woman is unknown. The frequency of
the allele is $0.00001$ in the population. 
{\bf A) } If they have a child what is the probability that the
child will be homozygote for the disease allele?  
{\bf B)} How would your answer change if they were 2nd cousins)? (2nd
cousins share a pair of great grandparents).
\end{question}

\begin{question}
You sequence a genomic region of a species of Baboon. Out of 100
thousand basepairs on average 2000 differ between each pair of
sequences. Assume a per base mutation rate of $1 \times 10^{-8}$ and a
generation time of ten years. \\
{\bf A)} What is the effective population size of these Baboons? \\
{\bf B)} What is the average coalescent time (in years) of a pair of
sequences in this species?
\end{question}

\begin{question}
You find a pair of sites where Neanderthal and human populations had a
fixed difference at both loci (all neanderthals had one allele and all
humans had the alternate allele). You find that in present day populations the frequency of haplotype with the
Neanderthal allele at both sites is 2.24\%. 
Based on independent information you
know that the initial Neanderthal admixture proportion was 5\%. The marginal frequency of the Neanderthal allele has not changed from
this initial frequency to be present day, at either locus. 
The recombination fraction between your loci is 0.0005.\\
How many generations back do you estimate that the Neanderthal
admixture occurred?   
\end{question}

\begin{question}
In a species of lemurs you estimate the allele frequency to be
$20\%$. In a particular sub-population you estimate that the allele
frequency is $10\%$. In this population only $9\%$ of individuals are
heterozygote. What is $F_{IT}$, $F_{ST}$, and $F_{IS}$?   
\end{question} 


\begin{question}
Under a model of (only) genetic drift why does the expected level of heterozygosity decrease by a factor of
$1-\nicefrac{1}{(2N)}$ every generation? Explain your answer in terms
of the probability of identity by descent.
\end{question}

\begin{question}
Why is the historical rate of genetic drift often much higher than the
census size of the population might suggest? Name two distinct
processes that can contribute to this mismatch in your answer.
\end{question}

\end{document}