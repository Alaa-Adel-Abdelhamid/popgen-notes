\documentclass[12pt]{article}
\usepackage{nicefrac}
\newcounter{question}[section]   %%modified from https://www.sharelatex.com/learn/Counters
%\newenvironment{question}[1][]{\refstepcounter{question}\par   \begin{tcolorbox}
 %   \medskip \textbf{Question~\thequestion. #1}\rmfamily}{\medskip} \end{tcolorbox}


\newenvironment{question}[1][]{\refstepcounter{question}\par\medskip
   \textbf{Question~\thequestion. #1} \rmfamily}{\medskip} 


\begin{document}
\title{EVE102 Practice Problems 3}
\author{Graham Coop}
\date{}
\maketitle

These study questions are to help you prepare for the exams.
These, and questions given in the homework, are the sorts of questions
that you can expect to see. Indeed, some of these questions are old
exam questions.  Your answers should not to be handed in and no key
will be posted.  If you have questions come to see your TA or
instructor during office hours or try asking your peers for help in
the Chat Room on Piazza.\\
 In the past students have worked
collaboratively on an answer key. We think that's great! However, remember to
attempt the questions multiple times before looking at the answers.\\

Remember the exams are cumulative, so please review your previous practice
problems, questions from the notes, and the first two midterms.

%Note that population genetics is an analytic
%subject that requires repeatedly working through the problems to fully
%understand it. 

{\bf Unless otherwise stated assume that loci are biallelic and
autosomal. \\
For full credit on the exam you must show your work, and state the asumptions you have to make
(if any). }\\

\begin{question} 
Why is the level of constraint in functional regions potentially a
function of effective population size?
\end{question} 

\begin{question}
What is a `super gene'? And how and why do they arise? 
\end{question}

\begin{question} 
Outline two distinct short-term advantages of asexual reproduction.
\end{question} 

\begin{question} 
What is the C-value paradox and how do transposable elements offer a solution to it?
\end{question} 

\begin{question}
A mutation arises that lowers an individual's fitness by
$10^{-4}$. What is the probability that this mutation fixes in a
population with effective size $N_e=10^4$? what is the probability it
fixes in a population of effective size $N_e=10^3$? Describe why you expect this to be the case.
\end{question}


\begin{question}
Describe an example of selection below the level of the individual.
\end{question}

\begin{question} 
Outline two distinct evolutionary advantages of sexual reproduction
that could explain its long-term maintenance.
\end{question} 

\begin{question} 
Why are lower levels of diversity in regions of low recombination
potentially consistent with hitchhiking supressing diversity levels in species
with large population sizes, such as Drosophila?
\end{question} 

\begin{question}
What aspect of female meiosis makes it vunerable to selfish systems? 
\end{question}

\begin{question} 
Natural selection does not directly favor changes in mating preference
under the good genes model of sexual selection. Yet mating preference
does evolve as an indirect response to selection on male
traits. Explain how can we understand this indirect response in a
quantative genetics framework, and why this indirect response occurs. 
\end{question} 

\begin{question} 
Why do Y chromosomes often have low gene content compared to X chromosomes? 
\end{question} 

\begin{question} 
Generation time and effective population size are inversely correlated
among species in nature.  How does this observation help us explain the protein molecular clock measuring time in years, despite mutation rates being relative constant across a range of different generation times?
\end{question} 


\begin{question} 
Briefly outline an evolutionary explanation for the widespread observation of a 50/50 sex ratio.
\end{question} 

\begin{question} 
Why can the shutting down for recombination between X and Y
chromosomes be initially selected for? 
\end{question} 


\begin{question}
Briefly outline why a male-biased sex ratio might evolve. Would this
be stable over long evolutionary time periods?
\end{question}

\begin{question}
An allele at the gene dhfr is thought to have recent swept malaria ({\it
  P. falciparum}) populations conferin anti-malaria drug resistance.  Random “facts” about P. falciparum, only some of which Graham made up (but all of which should be assumed to be true for the purposes of answering these questions). Recombination rates in P. falciparum are roughly 1cM/10kb. Drug resistance is thought to carry a fitness cost to when it is present in areas where anti-malaria drugs are not being administered. It has been estimated that in drug-free areas drug-resistant P. falciparum have a viability that is 90\% of that of nondrug-resistant P. falciparum.
The effective population size of P. falciparum is ~10,000.\\
{\bf A)} It takes roughly ~50kb for levels of diversity to return to
roughly 50\% of their genome-wide average as we move away from the
dhfr gene. Recombination rates in P. falciparum are roughly
1cM/10kb. Roughly what is the selection coefficient in favor of the
drug resistance allele?\\
{\bf B)}	You survey the frequency of drug resistance in the
malaria in a drug-free area (pop1) nearby to an area where drugs are
in use (pop2). You find the drug resistance allele to be at a frequency of 10\% in pop1. What is the migration rate between pop1 and pop2? \\
{\bf C} Imagine an isolated population where anti-malaria drugs have been heavily used and as a result the drug resistance allele is fixed. Drug use is stopped, and a reversion mutation, which is non-drug resistant, arises in the population. How long will it take to sweep to fixation?
\end{question}

\end{document}
