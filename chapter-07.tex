\section{Genetic drift and Neutral alleles}



\subsection{The fixation of neutral alleles}
It is very unlikely that a rare neutral allele accidentally drifts up
to fixation, it is much more likely that such an allele is eventually
lost from the population. However, there is a large and constant influx of
rare alleles into the population due to mutation, so even if it is very
unlikely that an individual allele fixes within the population, some
neutral alleles will fix.  \\

%We'll first consider the probability that a neutral allele fixes
%within the population, starting from it just enters a diploid
%population as a newly mutated allele at frequency $1/(2N)$.

%so for an allele to be fixed in the population it
%must have been that allele


\paragraph{Probability of the eventual fixation of a neutral allele.} An allele which reaches fixation within a population, is an ancestor to
the entire population. In a particular generation there can be only single
allele that all other alleles at the locus in  later generation can claim as an
ancestor. As at a neutral locus all of our alleles are exchangeable, as
they have no effect on the number of descendents an individual
leaves, so any allele is equally likely to be the ancestor of the
entire population.  In a diploid population size of size $N$, there are $2N$
alleles all of which are equally likely to be the ancestor of the
entire population at some later time point. So if our allele is present in a single copy, the chance that
is the ancestor to the entire population in some future generation is
$1/(2N)$, i.e. the chance our neutral allele is eventually fixed is
$1/(2N)$.\\

More generally if our neutral allele is present in $i$ copies in the
population, of $2N$ alleles, the probability that this allele is fixed
is $i/(2N)$. I.e. the probability that a neutral allele is eventually
fixed is simply given by its frequency ($p$) in the population.
We can also derive this result by letting $Ns \rightarrow
0$ in eqn. \eqref{eqn:prob_fixed}.

\paragraph{Rate of substitution of neutral alleles.}

A substitution between populations that do not exchange gene flow is
simply a fixation event within one population. The rate of
substitution is therefore the rate at which new alleles fix in the
population, so that the long-term substitution rate is the rate at
which mutations arise that will eventually become fixed within our population.\\

Assume that there are two classes of mutational changes that can occur with a
region, highly deleterious mutations and neutral mutations. A fraction
$C$ of all mutational changes are highly deleterious, and can not
possibly contribute to substitution nor polymorphism (i.e. $Ns \gg 1$).
While a fraction $1-C$ are neutral. If our mutation rate is $\mu$ per
transmitted allele per generation, then a total of $2N \mu (1-C)$
neutral mutations enter our population each generation.\\

Each of these neutral mutations has a $1/(2N)$ probability chance of
eventually becoming fixed in the population. Therefore, the rate at
which neutral mutations arise that eventually become fixed within our
population is  
\begin{equation}
2N\mu(1-C)\frac{1}{2N} = \mu(1-C)
\end{equation}
thus the rate of substitution under a model where newly arising alleles are either
highly deleterious or neutral, is simply given by the mutation rate
towards neutral alleles, i.e. $\mu(1-C)$.\\

Consider a pair of species have diverged for $T$ generations, i.e. orthologous sequences shared between the species last shared a common ancestor $T$ generations ago. If they have maintained a constant $\mu$ over that time, will have accumulated an average of
\begin{equation}
2\mu(1-C)T
\end{equation}
neutral substitutions. This assumes that $T$ is a lot longer than the time it
takes to fix a neutral allele, such that the total number of 
alleles introduced into the population that will eventually fix is the
total number of substitutions. We'll see below that a neutral allele
takes on average $4N$ generations to fix from its introduction into
the population.\\

This is a really pretty result as the population size has completely
canceled out of the neutral substitution rate. However, there is
another way to see this in a more straightward way. If I look at a
sequence in me compared to say a particular chimp, I'm looking at the mutations
that have occurred in both of our germlines since they parted ways $T$
generations ago. Since neutral alleles do not alter the probability
of their transmission to the next generation, we are simply looking at
the mutations that have occurred in $2T$ generations worth of
transmissions. Thus the average number of neutral mutational
differences separating our pair of species is simply $2\mu (1-C) T$.\\




\subsection{Comparing polymorphism and divergence}


\subsection{Deviations from the constant population model.}
We've seen previously that changes in our population size can be
captured by an effective population size. However, this will only be a
useful measure if population sizes vary rapidly enough, that the
harmonic mean effective population size over short time periods ($\ll
N_e$ generations) is representative of the effective population size averaged over
longer time periods. If this is not the case there is no one effective
population size, as we can not approximate our rate of drift by a
single constant population. Furthermore, we've ignored the effect of
population structure and selection which will violate our modeling
assumptions. \\

We can hope to detect violations from our constant population size
neutral model, by comparing aspects of our dataset to their expectations
and distributions under our neutral model. \\

For example we have devised two estimates of $\theta$,
$\widehat{\theta_{\pi}}$ and $\widehat{\theta_{W}}$, using
expectations of different aspects of our data (pairwise diversity and
number of segregating sites respectively). Under our constant neutral
model if we have sufficient data those two estimates should be
equal to each other on average. But if there's some violation of our model they might not
be. So one test statistic might be to take
\begin{equation}
D = \widehat{\theta_{\pi}} - \widehat{\theta_{W}}
\end{equation}
which will be zero in expectation if our data was generated by a
neutral constant population model.




\newpage
