<source-file filename="assignment_methods.tex" project="Assignment_methods">

\documentclass[12pt]{article}
\begin{document} 
\title{PBGG Notes 2011: The use of assignment methods in population genetics.}
\author{Graham Coop$^{1}$ \\
\small $^1$ Department of Evolution and Ecology \& Center for Population Biology,\\
}
\date{}
\maketitle

Here I describe a simple probabilistic assignment to find the probability that an individual of unknown population comes from one of $K$ predefined populations. I then briefly explain how to extend this to cluster individuals into $K$ initially unknown populations. This method is a simplified version of what Bayesian population genetics clustering algorithms such as STRUCTURE (Pritchard et al. Genetics 2000). 

\paragraph{A simple assignment method}

We have genotype data from unlinked S bi-allelic loci for $K$ populations. The allele frequency of allele $1$ at locus $l$ in population $k$ is denoted by $p_{k,l}$, so that the allele frequencies in population 1 are $p_{1,1},\cdots p_{1,L}$ and population 2 are $p_{2,1},\cdots p_{2,L}$ and so on. 

You type a new individual from an unknown population at these $L$ loci. This individual's genotype at locus $l$ is $g_l$, where $g_l$ denotes the number of copies of allele 1 this individual carries at this locus $g_l=0,1,2$). 

The probability of this individual's genotype at locus $l$ conditional on coming from population $k$ (i.e. their alleles being a random HW draw from population $k$) is 
\begin{equation}
P(g_l | \textrm{pop k}) = I(g_l=0) (1-p_{k,l})^2 +  I(g_l=1) 2 p_{k,l} (1-p_{k,l}) + I(g_l=2) p_{k,l}^2
\end{equation}
where $I(g_l=0)$ is an indicator function which is $1$ if $g_l=0$ and zero otherwise, and likewise for the other indicator functions. 

Therefore, assuming that the loci are independent, the probability of individual's genotypes conditional on them coming from population $k$ is 
\begin{equation}
P(\textrm{new ind.} | \textrm{pop k})  = \prod_{l=1}^S P(g_l | \textrm{pop k}) \label{eqn_assignment}
\end{equation}

We wish to know the probability that this new individual comes from population $k$, i.e. $P(\textrm{pop k} | \textrm{new ind.})$. We can obtain this through Bayes rule 
\begin{equation}
 P(\textrm{pop k} | \textrm{new ind.})  = \frac{P(\textrm{new ind.} | \textrm{pop k}) P(\textrm{pop k})}{P(\textrm{new ind.})}
\end{equation}
where 
\begin{equation}
P(\textrm{new ind.}) = \sum_{k=1}^K  \frac{P(\textrm{new ind.} | \textrm{pop k}) P(\textrm{pop k})}{P(\textrm{new ind.})}
\end{equation}
is the normalizing constant. We interpret $P(\textrm{pop k})$ as the prior probability of the individual coming from population $k$, unless we have some prior knowledge we will assume that the new individual has a equal probability of coming from each population $P(\textrm{pop k})=1/K$.  

We intepret 
\begin{equation}
 P(\textrm{pop k} | \textrm{new ind.})
\end{equation}
as the posterior probability that our new individual comes from each of our $1,\cdots, K$ populations.

More sophisticated versions of this are now used to allow for hybrids, e.g, we can have a proportion $q_k$ of our individual's genome come from population $k$ and estimate the set of $q_k$'s.
\paragraph{clustering based on assignment methods}
We wish to cluster our individuals into $K$ unknown populations. We begin by assigning our individuals at random to these $K$ populations. 
\begin{itemize}
\item Given these assignments we estimate the allele frequencies at all of our loci in each population. 
\item Given these allele frequencies we chose to reassign each individual to a population $k$ with a probability given by eqn. ($\ref{eqn_assignment}$).
\end{itemize}
We iterate steps 1 and 2 for many iterations. If the data is sufficiently informative the assignments and allele frequencies will quickly converge. 

Technically this is the joint posterior of our allele frequencies and assignments. 

\end{document}


</source-file>
